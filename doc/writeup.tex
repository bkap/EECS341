\documentclass[11pt,oneside,a4paper]{article}
\usepackage{textcomp}
\usepackage{setspace}
\usepackage{graphicx}
\usepackage{amsmath}
\title{EECS 341 Database Project: Class Scheduler}
\begin{document}
\maketitle
\tableofcontents
\section{Introduction}
This project is a course scheduling program for students at a University
(similar to the Student Information System that Case uses, but one that doesn't
take as long to execute queries).
\section{Application Requirements Specifications}
The application has three classes of users: Students, Professors, and
Administrators. Each has a set of actions that they will be able to perform.
\subsection{Administrators}
Administrators are able to add new users, define new semesters, create new
courses and classes, view and set the grades of any student, and enroll students in
classes.
\subsection{Professors}
Professors are capable of being listed as teaching classes. They are able to
view and set the grades of the students for the classes taught by the
professor. They are capable of enrolling students in classes that they teach.
\subsection{Students}
Students are capable of enrolling in classes, provided the class is not at its
maximum capacity, and viewing their own grades.
\subsection{All users}
All users can search the database of classes. They can see basic information
about the class as well as viewing who is enrolled in each class.

\section{Database Requirements Specification}
\subsection{Tables}
User(\underline{username: varchar(30)},firstname: varchar, lastname: varchar,
password: Pass	word, usertype: Enum(Student,Professor,Admin))
\\
Semester(\underline{name: varchar},start\_date:date,end\_date:date,reg\_start\_date:date,reg\_end\_date:date)
\\
Course(\underline{courseID	: Integer},name:varchar,dept:varchar(4), number:
SmallInt, prereq: Integer (ForeignKey to Course), description: text)
\\
Class(\underline{classNumber: Integer}, semester: varchar (ForeignKey to
Semester), days\_met : varchar(5), start\_time\_met : Time, end\_time\_met:
Time, course: Integer (ForeignKey to Course), professor: varchar(30)
(ForeignKey to User), max\_class\_size: Integer, buildingName: varchar,
roomNum: Integer)
\\
EnrolledClass(\underline{student: varchar(30) (Foreign Key to User), class\_enrolled : Integer (Foreign Key to Class)}, grade: SmallInt)
\\
Schedule(\underline{user: varchar(30) (ForeignKey to User), semester: varchar(ForeignKey to Semester), class\_enrolled: Integer(ForeignKey to Class)})
\\
CoursesTaken(\underline{student: varchar(30) (ForeignKey to User), courseID: Integer}) \\
Room(\underline{buildingName: varchar, roomNumber: Integer}, max\_capacity:
Integer)
\subsection{Queries and Transactions}
Admins need to be able to add users and classes. Students need to be able to
view their schedules (query Schedule based on user and semester), enroll in classes (Create EnrolledClass and Schedule
objects). They also need to be able to view their grades. Professors need to be
able to set their students grades for each class they teach.

Adding users and classes will only occur twice a year, and will probably be
done in bulk.
	
\section{ER data model design}
List the entities and their attributes. Specify the domain of each attribute. Specify the properties of each attribute (i.e., key, foreign key, composite/simple, single-valued/multi-valued, derived, incomplete with different nulls, roles, weak/strong entity type, etc.). Specify the relationships and their attributes. List the properties of each relationship (i.e., degree of the relationship, cardinality ratios (1-1, 1-N, N-M), participation constraints (total, partial), other application-specific constraints, etc.). Your ER model should have at least 5 entity types, 7 relationship types, and sufficiently many attributes and constraints of different types. Draw the E-R diagram, and incorporate everything discussed in this step into the E-R diagram. 

\subsection{Entities}
The ER diagram is attached. There are 7 Entities in our model: User, Class, Course, Semester, Schedule, Room, Department.  They are listed below in the following form:
\\
entityName(\underline{primaryKeyAttribute: attributeType}, attributeName: attributeDomain, foreignKeyAttribute: foreignKeyDomain(Foreign Key to Entity))
\\
User(\underline{username: varchar(30)},firstname: varchar, lastname: varchar,
password: Pass	word, usertype: Enum(Student,Professor,Admin))
\\
The User has the following attributes: username, firstname, lastname, 
password, usertype.encompases students, professors and administrators.  
Each user will have a first and last name. Since logging into the system 
will require authentication there will be a password for each user.
\\
Class(\underline{classNumber: Integer}, semester: varchar (ForeignKey to
Semester), days\_met : varchar(5), start\_time\_met : Time, end\_time\_met:
Time, course: Integer (ForeignKey to Course), professor: varchar(30)
(ForeignKey to User), max\_class\_size: Integer, buildingName: varchar,
roomNum: Integer)
\\
Course(\underline{courseID	: Integer},name:varchar,dept:varchar(4), 
number: SmallInt, prereq: Integer (ForeignKey to Course), description: text)
\\
Semester(\underline{name: varchar},start\_date:date,end\_date: date, 
reg\_start\_date: date,reg\_end\_date:date)
\\
Room(\underline{buildingName: varchar, roomNumber: Integer}, max\_capacity:
Integer)\\
Department(\underline{name: varchar(4)}, professor: User(foreign key to user))
\subsection{Relations} 
Each entity is related to several others in our model.  This section lists and describes 
all of the relations as well as their cardinality. 
\\
\subsubsection{User has Schedules (1:N)}
Each user can have a schedule if they are a student.  this relation "has" does not have
 any attribues. The cardinality is one to many because a stduent(user) can have 1 
schedule for each semester. student's do not have to have any schedules.  If they have not
 registered for any classes they will not yet have a schedule.  In this case, we need to 
ensure that there is a constraint on the database that does not show the schedule of the 
student if they do not yet have one.  
\subsubsection{User is a part of Department (1:1)}
If a user is a teacher (or student?) then they may be a part of the a department.
 The cardinality of this relation is one to one; each teacher may only belong to 
one department.
\subsubsection{user attends/teaches/creates class N:M}
All users will will be related to teachers based on their type.  Student's enroll and 
attend classes, Teachers teach classes, and administrators create classes.  Any users 
of these three types can be associated with multiple classes additionally any class 
can be assosiciated with multiple users, therefore the cardinality is many to many. 
Finally this relation has an attribute: grade.  Grades are given to students who 
finish the class. This attribute is assigned by teacher.
\subsubsection{Schedule contains Classes (N:M)}
\\
A schedule is a list of classes that a user is enrolled in for a specific semester.  Each schedule can contain multiple classes.  Each class can be in multiple schedules. Therefore, the cardinality is many to many. 
\subsubsection{Class in a Room (N:1)}
Each class must be held in a Room.  The cardinality of this relationship is many to one.  The reasoning for this is that each room can have multiple classes. However, each class is distinguished by its meeting place and time in additionto other variables.  So there can only be one room for each class. 
\section{ER Integrity Constraints}
\subsection{Constraints}
\begin{itemize}
	\item Class.professor.usertype = Professor. Will happen pretty
	frequently- every time classes are created for the new semester\\
	\item on create EnrolledClass, EnrolledClass.student.usertype = Student. Will
	happen frequently- for each class each student goes to enroll in.\\
	\item on create Schedule, ensure Schedule.class\_enrolled.semester =
	Schedule.semester\\
	\item On create EnrolledClass, if class\_enrolled.course.prereq is not NULL, ensure
	that there exists (SELECT * FROM CoursesTaken T where T.user =
	EnrolledClass.user and T.courseID =
	class\_enrolled.course.prereq.courseID)\\
	\item On create Class, if buildingName and roomNumber are not NULL, ensure that
	there does not exist a Class in the same room at the same time in the same
	semester, and ensure that Class.max\_class\_size <= Room.max\_capacity \\
	\item On create EnrolledClass, ensure that there does not exist a Schedule such
	that user1 = user2, semester1=semester2, and the class meeting times
	overlap

\end{itemize}

The Foreign Keys will be enforced through the Database. We are using an
Object-Relational Manager so it takes care of these for us.

All of the other constraints will be enforced through the application's code. This is especially important because
Admins need the ability to violate some of these constraints (for instance, to
enroll students in courses that are already at max\_capacity)
\subsection{Triggers}
	* On create EnrolledClass,  add Schedule (user, class.semester,
	EnrolledClass). 

\section{Queries}
\subsection{Query 1}
 SELECT * FROM Class WHERE class.professor = professor AND class.semester =
 semester AND EXISTS (Select * from course where course.id = class.course and
 course.dept = dept andcourse.number $<$ number) 

RA: 
\begin{align*}(\sigma_{class.professor = professor} (\sigma_{class.semester = semester}
Class)) \bowtie_{class.course = course.id} \\
\sigma_{course.number < number}
\sigma_{course.dept = dept} Course 
\end{align*}
RC: 
\begin{align*} \{t | \exists C \in Class ( t = C \wedge C.professor = professor \wedge
c.semester = semeser \wedge \\
\exists C2 \in Course ( c2.id = c.course \wedge \\
c2.dept = dept \wedge c2.number < number))\}\end{align*}

\subsection{Query 2}
(For this query, class, user, and today are parameters of the query chosen by
the user) \\

\texttt{SELECT * FROM Class WHERE Class.id = class AND EXISTS (SELECT * FROM
Semester WHERE Semester.name = Class.semester AND Semester.reg\_start\_date $<$
today AND Semester.reg\_end\_date $>$ today) AND NOT EXISTS (SELECT * FROM
EnrolledClass WHERE EnrolledClass.class = class.id GROUP BY EnrolledClass.class
HAVING Count(*) $>$ Class.max\_capacity) AND NOT EXISTS (SELECT * FROM
EnrolledClass EC2,Class C2 WHERE EC2.user = user AND C2.id = EC2.class AND
C2.semester = Semester.name AND C2.days\_met = Class.days\_met AND
(C2.start\_time $<$ Class.start\_time AND C2.end\_time $>$ Class.start\_time)
OR (C2.start\_time $>$ Class.start\_time and C2.start\_time $<$
Class.end\_time)) AND EXISTS (SELECT * FROM Course where Course.id =
Class.course AND (Course.prereq = NULL OR EXISTS (SELECT * From EnrolledClass
ec3, Class c3 WHERE ec3.user = user and ec3.class = c3.id AND c3.course =
Course.prereq)))} 

RA:
\begin{align*} 
&\sigma_{class.id=\textbf{class}}(Class) \bowtie \\
& \sigma_{reg\_start\_date  < \textbf{today} \wedge reg\_end\_date >
\textbf{today}}(Semester) - \\
&(Class \bowtie \pi_{class}(\sigma_{\,_{class}G_{Count(*)} >
maxcapacity}(EnrolledClass) \cup \\
&\sigma_{\textbf{class}.days\_met = days\_met} \\
&\sigma_{ 
(start\_time < \textbf{class}.start\_time \wedge end\_time >
\textbf{class}.start\_time)
\vee (start\_time > \textbf{class}.start\_time 
\wedge start\_time > \textbf{class}.end\_time)} \\
&(Class \bowtie \\
&\sigma_{ec.user = \textbf{user}\vee 
ec.semester=\textbf{semester}}(EnrolledClass)))) 
\end{align*}

RC:
\begin{align*} \{ t | & \exists C, Co | \{Class(C) \wedge C.id = t.id
  \wedge \\
 &  Course(Co)  \wedge Co.id = C.course \wedge \\
 &  \exists S | \{Semester(S) \wedge S.name = C.semester \wedge \\
&   S.reg\_start\_date < today \wedge S.reg\_end\_date > today\} \wedge \\
 &  \neg \exists EC2, C2 | \{EnrolledClass(EC2) \wedge Class(C2) \wedge\\
&   EC2.user = user \wedge C2.days = C.days \wedge \\
&   ((C2.start\_time < C.start\_time \wedge C2.end\_time > C.start\_time ) \vee \\
&  (C2.start\_time > C.start\_time \wedge C2.start\_time < C.end\_time))\} \wedge
  \\
&   COUNT(E | \{EnrolledClass(E) \wedge E.class = C.id \}) < C.max\_capacity \wedge \\
&  (co.prereq = NULL \vee \exists EC3, C3 | \{EnrolledClass(EC3) \wedge \\
&   Class(C3)
  \wedge EC3.user = user \wedge EC3.class = c3.id \wedge C3.course =
  co.prereq\})\}\}
\end{align*}

\subsection{Query 3}
\texttt{SELECT Class.name, Class.dept, Class.number, Class.start\_time\_met,
Class.end\_time\_met, Class.day\_met FROM Class, Semester WHERE Class.semester
= Semester.name AND Semester.start\_date $<$ \textbf{today} and
Semester.end\_date $>$ \textbf{today} AND EXISTS (Select * From EnrolledClass
WHERE EnrolledClass.user = \textbf{user} AND EnrolledClass.class = Class.id)}


RA:
\begin{align*}
\sigma_{user = \textbf{user}}(EnrolledClass) \bowtie \\
\sigma_{class.semester = \textbf{semester}} Class \bowtie \\
\sigma_{semester.start\_date < \textbf{today}} \wedge semester.end\_date > \\
\textbf{today}(Semester) \end{align*}

RC:
\begin{align*}
\{t | & \exists C,S | \{Class(C) \wedge Semester(S) \wedge C.semester = S \\
& \wedge Semester.start_date < \textbf{today} \wedge Semester.end_date > \\
& \textbf{today} \wedge \exists E | \{EnrolledClass E \wedge E.user = \\
&  \textbf{user} \wedge E.class_enrolled = C.id\}\}\}
\end{align*}
\subsection{Semester}
\hspace{0.5in}name$\rightarrow$ start\_date, end\_date, reg\_start\_date, reg\_end\_date  
\section{Functional Dependencies and 3NF}

\subsection{Room}
\hspace*{0.5in}buildingname, roomnum $\rightarrow$ max\_capacity


\subsection{Course}
\hspace*{0.5in}dept, number$\rightarrow$name, prereq, description \\
\hspace*{0.5in}dept$\rightarrow$id \\
\hspace*{0.5in}id$\rightarrow$(everything\_else) 


\subsection{Class}
\hspace*{0.5in}id$\rightarrow$semester, days\_met, start\_time\_met, end\_time\_met, class\_size,course, professor, room, max\_class\_size \\
\hspace*{0.5in}semester, room, start\_time\_met, days\_met $\rightarrow$(everything else) \\
\hspace*{0.5in}semeseter, room, end\_time\_met, days\_met$\rightarrow$(everything else) \\
\hspace*{0.5in}semester, professor, start\_time\_met, days\_met$\rightarrow$(everything else) \\
\hspace*{0.5in}semester, professor, end\_time\_met, days\_met$\rightarrow$(everything else) 


\subsection{EnrolledClass}
\hspace*{0.5in}student,class\_enrolled $\rightarrow$grade


\subsection{Schedule}
\hspace*{0.5in}user,semester$\rightarrow$schedule\_id 


\subsection{Schedule\_class\_enrolled}
\hspace*{0.5in}schedule\_id,class$\rightarrow$student \\
\hspace*{0.5in}class,student$\rightarrow$schedule\_id 

For each of these functional dependencies X$\rightarrow$A, X is a candidate
key. Therefore, the document is in BCNF.
\section{Revising the Schema}
After implementing the program, we added the Schedule table. As mentioned
before, this table is defined as 
\begin{quote}
Schedule(\underline{user: varchar(30) (ForeignKey to User),} \\
\underline{semester: varchar (ForeignKey to Semester),}\\
\underline{class\_enrolled: Integer(ForeignKey to Class)})
\end{quote}

This schedule is used to optimize several of our queries. First of all, it
allows us to quickly find classes by User and Semester. This is used by two of
the more complicated queries: finding a student's classes by semester (for the
schedule and for the grades), and for checking for scheduling conflicts when
the student tries to enroll in another class.

We're also looking at adding the CoursesTaken table. This table allows us to
check for prereqs with just a single SELECT query. WIthout this table, we'd
have to join EnrolledClass with Class and then join that with Course.
\section{Revisting the Whole Project}
A lack of time with other classes made this project somewhat rushed. My group agrees that we would have liked to make the system more verbose.  While the html pages work just fine we would have like to have the time to creat Casscading style sheets and other web formatting to make the web application much prettier.  Additional resources would have made this much easier.  The system as is works for what we intended.  It is somewhat simple but could be expanded with signifigant work into a commonly used application.
\section{Team Work}
All team members work effectively and effieciently together. 
\section{Conclusions}
Appendicies and diagrams reffered to in this Document are attached. Please contact me if you have any questions with this report.  
\end{document}

