\documentclass[11pt,oneside,a4paper]{article}
\usepackage{textcomp}
\usepackage{setspace}
\usepackage{graphicx}
\usepackage{amsmath}
\title{EECS 341 Database Project: User Manual}
\begin{document}
\maketitle
\section{Introduction}
Enclosed is a short maunal for the average user of this application.  
Thise maunal is broken into three sections based on the three types of 
users: Students, professors, and administrators.
\section{Administrators}
\subsection{Overview}
Administrators Administrators are able to add new users, define new semesters, create new
courses and classes, view and set the grades of any student, and enroll students in
classes.  All of this is simple to do from the administrator panel on the server.  
\subsection{Login}
The admin is located on the server on URL_of_the_server/admin. The first time you attempt
to login you will be prompted to create a new administrator.  Follow the instructions that 
do this.  Once you have logged into the system you may manually add students, professors, 
other administrators.  You may create Classses, Courses, Enrolled classses, Rooms, 
Schedules, and Semester.
\section{Professors}
\subsection{Login}
Professors may use this system to view the students which are enrolled 
in their classes and assign grades.The page will be in the form URL_of_the_Server/Scheduler/Login.html. 
 To login you simply fill in your credentials and click the login button; these will be 
given to you by your system administrator.  
\subsection{View Students}
Once you have logged into the system you will be redirected to a home page that lists all 
of the classes that you are scheduled to teach. there is a button by each of these classes
 which says view student list.  Pressing this button will redirect you to the list of 
students for this class.
\subsection{Set Grades}
Grades may be set seperately using the drop down menus provided.  Clicking submit will save
 them to the database.  They may be updated later.
\section{Students}
\subsection{Login}
The primary use of this system for students is to enroll and withdraw in classes.  
The system which is simple and intuitive, is viewed through a series of simple html webpages 
which students can explore to search for and enroll in classes.  For a first time user you 
simeply go to the login page.  The page will be in the form URL_of_the_Server/Scheduler/Login.html. 
 To login you simply fill in your credentials and click the login button; these will be 
given to you by your system administrator.  
\subsection{HomePage}
Once you have logged into the system you will be redirected to a Home Page.  Thise page 
will list all of the classes that you are currenly enrolled in.  From this page there are 
multiple options. First you may Search from classes (by clicking on the Search button) 
second you may view your grades ( by clicking the view grades button).
\subsection{Search for Classes}
The Class Search form has a few options that you can use. Onece you have filled out the 
form you may click the search button alternatively pressing enter will also work. Once
You have made a search you will be redirected to a page with all of the classes that 
fit into your schedule.  
\subsection{Enrolla Class}
Upon searching for classes using the search function; you will be givin a list of classes which
fit the criteria you set.  You will be able to select an Enroll button adjacent to these classes
to enroll in them.  You will be unable to enroll in a class that conflicts with any class you are
 currently registered for. 
\subsection{View Grades for a Class}
From the home page there is a view grades button. Selecting this will bring you to another page
 which allows you to chose a semester.  Hit the get grades button an classes from that semester
 that you have been given grades will be visible.
\end{document}
