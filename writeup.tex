\documentclass[11pt,oneside,a4paper]{article}
\usepackage{textcomp}
\usepackage{setspace}
\usepackage{graphicx}
\title{EECS 341 Database Project: Class Scheduler}
\begin{document}
\maketitle
\section{Introduction}
This project is a course scheduling program for students at a University
(similar to the Student Information System that Case uses, but one that doesn't
take as long to execute queries).
\section{Application Requirements Specifications}
The application has three classes of users: Students, Professors, and
Administrators. Each has a set of actions that they will be able to perform.
\subsection{Administrators}
Administrators are able to add new users, define new semesters, create new
courses and classes, view and set the grades of any student, and enroll students in
classes.
\subsection{Professors}
Professors are capable of being listed as teaching classes. They are able to
view and set the grades of the students for the classes taught by the
professor. They are capable of enrolling students in classes that they teach.
\subsection{Students}
Students are capable of enrolling in classes, provided the class is not at its
maximum capacity, and viewing their own grades.
\subsection{All users}
All users can search the database of classes. They can see basic information
about the class as well as viewing who is enrolled in each class.

\section{Database Requirements Specification}
\subsection{Tables}
User(\underline{username: varchar(30)},firstname: varchar, lastname: varchar,
password: Pass	word, usertype: Enum(Student,Professor,Admin)
\\
Semester(\underline{name:
varchar},start\_date:date,end\_date:date,reg\_start\_date:date,reg\_end\_date:date)
\\
Course(\underline{courseID	: Integer},name:varchar,dept:varchar(4), number:
SmallInt, prereq: Integer (ForeignKey to Course), description: text)
\\
Class(\underline{classNumber: Integer}, semester: varchar (ForeignKey to
Semester), days\_met : varchar(5), start\_time\_met : Time, end\_time\_met:
Time, course: Integer (ForeignKey to Course), professor: varchar(30)
(ForeignKey to User), max\_class\_size: Integer, buildingName: varchar,
roomNum: Integer)
\\
EnrolledClass(\underline{student: varchar(30) (Foreign Key to User),
class\_enrolled : Integer (Foreign Key to Class)}, grade: SmallInt)
\\
Schedule(\underline{user: varchar(30) (ForeignKey to User), semester: varchar
(ForeignKey to Semester), class\_enrolled: Integer(ForeignKey to Class)}) <-
Join table, to get the list of classes the user is enrolled in for a specific
semester
\\
CoursesTaken(\underline{student: varchar(30) (ForeignKey to User), courseID:
Integer}) <-- a list of the clases the student has taken\\
Room(\underline{buildingName: varchar, roomNumber: Integer}, max\_capacity:
Integer)
\subsection{Constraints}
	* on create Class, Class.professor.usertype = Professor. Will happen pretty
	frequently- every time classes are created for the new semester\\
	* on create EnrolledClass, EnrolledClass.student.usertype = Student. Will
	happen frequently- for each class each student goes to enroll in.\\
	* on create Schedule, ensure Schedule.class\_enrolled.semester =
	Schedule.semester\\
	* On create EnrolledClass, if class\_enrolled.course.prereq is not NULL, ensure
	that there exists (SELECT * FROM CoursesTaken T where T.user =
	EnrolledClass.user and T.courseID =
	class\_enrolled.course.prereq.courseID\\
	* On create Class, if buildingName and roomNumber are not NULL, ensure that
	there does not exist a Class in the same room at the same time in the same
	semester, and ensure that Class.max\_class\_size <= Room.max\_capacity
\subsection{Triggers}
	* On create EnrolledClass, if user already has class scheduled for that
	time (check Schedule), cancel the create. Otherwise, add Schedule (user, class.semester,
	EnrolledClass). 
\end{document}

